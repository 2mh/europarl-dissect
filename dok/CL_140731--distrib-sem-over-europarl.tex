%%%%%%%%%%%%%%%%%%%%%%%%%%%%%%%%%%%%%%%%%%%%%%%%%%%%%%%%%%%%%%%%%%%%%%%%
% -*- coding: utf-8 -*-
\documentclass[11pt,twoside,openright]{mpreport}
\usepackage[utf8]{inputenc}
\usepackage[T1]{fontenc}
\usepackage[french, russian, icelandic, english, ngerman]{babel}
\usepackage{pxfonts}
\usepackage[numbers]{natbib}
\usepackage{booktabs}
\usepackage{gb4e} %\noautomath % For linguistic examples %%%einfach muell
%\usepackage{listings}         % For listings %%%einfach muell
%\usepackage{color} %%%einfach muell
\usepackage{hyperref}         % Hyperref should normally be the last
                              % package
\usepackage{qtree}
\usepackage{rotating}
\usepackage{url}
\usepackage{spverbatim}
\usepackage{float}
\usepackage{longtable}
\usepackage{newunicodechar}
\brokenpenalty10000\relax

\let\cleardoublepage\clearpage

% Information for the title page
\usepackage[type=seminar,
            seminar={Fragestellungen der statistikbasierten Semantik},
            semester={Frühlingssemester 2014}]{cluzh-title-cd2010}

% Avoid widows and orphans
\widowpenalty50000
\clubpenalty50000

% URL with access date
\newcommand{\urld}[2][???]{\url{#2} (\iflanguage{ngerman}{Letzter
    Abruf:}{accessed} #1)}

% Examples for the definition of convenience commands
\newcommand{\first}[1]{\emph{#1}}
\newcommand{\q}[1]{\iflanguage{ngerman}{\flqq#1\frqq}{``#1''}}
\newcommand{\gloss}[1]{`#1'}
\newcommand{\example}[1]{\emph{#1}}

%%%%%%%%%%%%%%%%%%%%%%%%%%%%%%%%%%%%%%%%%%%%%%%%%%%%%%%%%%%%%%%%%%%%%%%%%%%%%%%%
%%%%%%%%%%%%%%%%%%%%%%%%%%%%%%%%%%%%%%%%%%%%%%%%%%%%%%%%%%%%%%%%%%%%%%%%%%%%%%%%
%%%%%%%%%%%%%%%%%%%%%%%%%%%%%%%%%%%%%%%%%%%%%%%%%%%%%%%%%%%%%%%%%%%%%%%%%%%%%%%%

% Title page information
\title{Distributionelle Semantik über Europarl-Korpora
}
\author{Reto Baumgartner \and Hernani Marques}

\begin{document}
\maketitle
\small
\tableofcontents
\normalsize

%%%%%%%%%%%%%%%%%%%%%%%%%%%%%%%%%%%%%%%%%%%%%%%%%%%%%%%%%%%%%%%%%%%%%%%
\chapter{Einleitung}
\label{cha:einleitung}
%%%%%%%%%%%%%%%%%%%%%%%%%%%%%%%%%%%%%%%%%%%%%%%%%%%%%%%%%%%%%%%%%%%%%%%
\chapter{Theoretische Basis}
\label{cha:theorie}
%%%%%%%%%%%%%%%%%%%%%%%%%%%%%%%%%%%%%%%%%%%%%%%%%%%%%%%%%%%%%%%%%%%%%%%
\chapter{Praxis}
\label{cha:praxis}
%%%%%%%%%%%%%%%%%%%%%%%%%%%%%%%%%%%%%%%%%%%%%%%%%%%%%%%%%%%%%%%%%%%%%%%
\chapter{Evaluation}
\label{cha:evaluation}
%%%
\section{Übersetzungskandidaten DE--EN mit 10'000 Trainingssätzen über Wortoberflächen}
\label{sec:evalDeEn10kSurf}
\section{Übersetzungskandidaten EN--DE mit 10'000 Trainingssätzen über Wortoberflächen}
\label{sec:evalDeEn10kSurf}
\section{Übersetzungskandidaten EN--DE mit 10'000 Trainingssätzen über Wortoberflächen}

%%%%%%%%%%%%%%%%%%%%%%%%%%%%%%%%%%%%%%%%%%%%%%%%%%%%%%%%%%%%%%%%%%%%%%%
\chapter{Schluss}
\label{cha:schluss}
%%%%%%%%%%%%%%%%%%%%%%%%%%%%%%%%%%%%%%%%%%%%%%%%%%%%%%%%%%%%%%%%%%%%%%%
\bibliographystyle{plain-de}     % For German
\bibliography{seminar}
\addcontentsline{toc}{chapter}{Literaturverzeichnis}
%%%%%%%%%%%%%%%%%%%%%%%%%%%%%%%%%%%%%%%%%%%%%%%%%%%%%%%%%%%%%%%%%%%%%%%
\listoffigures
\addcontentsline{toc}{chapter}{Abbildungsverzeichnis}
\end{document}
